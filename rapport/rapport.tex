\documentclass[11pt,a4paper]{article}
\usepackage[top=40px, left=60px, right=60px]{geometry}
\usepackage[T1]{fontenc}
\usepackage[normalem]{ulem}
\usepackage{verbatim}
\usepackage[utf8]{inputenc}
\usepackage[francais]{babel}
\usepackage{graphicx}
\usepackage{placeins}
\usepackage{amsfonts}
\usepackage{amssymb}
\usepackage{amsmath}
\usepackage{amsthm}

\title{Factorisation d'entiers}
\author{Stéphane Horte \& Gabriel Lewertowski}

\begin{document}
\maketitle

\tableofcontents

\begin{abstract}
Nous présentons trois algorithmes de factorisation d'entiers : les algorithmes \textit{$\rho$} et \textit{$(p-1)$} proposés par J.M Pollard en 1974 dans \cite{pollard_rho} ainsi que l'algorithme \textit{ECM} (Elliptic Curve Method) décrit par H. W. Lenstra, Jr dans \cite{lenstra}. Nous testons ensuite nos implémentations sur des grands nombres choisis aléatoirement, et nous comparons avec les résultats obtenus par GMP-ECM.
\end{abstract}

\begin{thebibliography}{9}
\bibitem{pollard_rho}
J. M. Pollard, \emph{A Monte Carlo method for factorization}, BIT, 1975, pp.331-334

\bibitem{lenstra}
H. W. Lenstra, Jr, \emph{Factoring integers with elliptic curves}, Annals of Mathematics, 1987, pp.649-673


\end{thebibliography}

\end{document}
